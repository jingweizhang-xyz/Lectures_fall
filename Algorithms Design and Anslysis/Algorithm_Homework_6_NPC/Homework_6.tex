\documentclass[]{article}
\usepackage{ctex}
\usepackage{amsmath}
\usepackage{amsthm}
\usepackage{amssymb}
\usepackage{listings}
\usepackage{clrscode3e}
\usepackage{algorithm}
\usepackage{algorithmic}
\usepackage{graphicx}
\usepackage[top=2cm, bottom=2cm, left=2cm, right=2cm]{geometry}
\renewcommand\qedsymbol{$\blacksquare$}

\title{Algorithm Homework 6 NP}
\date{2015-1-7}
\author{Jingwei Zhang 201528013229095}

\begin{document}
    \maketitle
    \section{Problem 1}
        \begin{proof}[proof: Integer-programming is NPC]
        \item Firstly, by checking $Ax \geq b$, we can verify this problem in polynomial time (matrix multiplication takes $O(mn^2)$ and comparison takes $O(m)$). Thus, Integer-programming problem is a NP problem.
        \item Secondly, we will prove that 3SAT problem, a known NPC problem, is reducible to Integer-programming problem in polynomial time. For any instance $I$ of 3SAT problem (its formula is $\phi$), which contains $m$ variables $x_i(i = 1,\dots,m)$ and $n$ clauses, we construct a instance $I_i$ of Integer-programming as follows: 
        \begin{align}
        y_i \geq 0 \quad & \text{for all }i = 1,\dots,m \label{Eq_g0} \\
        y_i \leq 1 \quad & \text{for all }i = 1,\dots,m \label{Eq_s1}\\
        T_i + T_j + T_k \geq 1 \quad & \text{for all clause } (X_i \vee X_j \vee X_k) \label{Eq_clauses}
        \end{align}
        where 
        \begin{align}
        T_i = \begin{cases}
                y_i \quad & \text{if } X_i = x_i \\
                1 - y_i\quad & \text{if } X_i = \neg x_i 
              \end{cases} 
        \end{align}
        The inequalities above is equivalent to $Ax \geq b$ since every inequality can be changed to $\sum_j a_{ij} x_j \geq b_i$. $y_i=1$ is equivalent to $x_i=true$ and $y_i=0$ is equivalent to $x_i= false$. Then, we will prove these inequalities(or this Integer-programming problem) is equivalent to 3SAT problem.
        \begin{itemize}
        \item Suppose there exits an assignment of variables $x_i(i = 1,\dots,m)$ such that $\phi$ is $true$. We assign $y_i$ with rules mentioned above. It is obvious that inequalities (\ref{Eq_g0}) and (\ref{Eq_s1}) are satisfied. Note that (\ref{Eq_clauses}) holds iff at least one of $X_i, X_j, X_k$ is true. Thus, all constrains will hold.
        \item Suppose there exits an assignment of variables $y_i(i = 1,\dots,m)$ such that inequalities (\ref{Eq_g0}), (\ref{Eq_s1}) and (\ref{Eq_clauses}) hold. It is obvious that $y_i$ equals to $0$ or $1$. We assign $y_i$ with rules mentioned above. Note that (\ref{Eq_clauses}) holds iff at least one of $X_i, X_j, X_k$ is true. Thus, $\phi$ is $true$.
        
        \end{itemize}
        \item Thus, Integer-programming problem is NPC.
        \end{proof}
        
        
        
    \section{Problem 3}
        %{Half-3AST is NPC:}
        \begin{proof}[proof: Half-3SAT is NPC]
        \item Firstly, by substituting the variables with the given assignment , then checking whether half of its clauses is true and half is false, we can verify this problem in polynomial time. Thus, Half-3SAT problem is a NP problem.
        \item Secondly, we will prove that 3SAT problem, a known NPC problem, is reducible to Half-3SAT problem in polynomial time. For any instance $I$ of 3SAT problem, which contains $m$ variables $x_i(i = 1,\dots,m)$ and $n$ clauses, we construct a instance $I_h$ of Half-3SAT as follows: 
        \begin{align}
       \phi_{Ih} &= \phi_I \wedge T \wedge D \wedge \dots \wedge D \; \text{(D is repeated for }n+1 \text{ times)}   \\
        D &= (x_{m+1} \vee x_{m+2} \vee x_{m+3}) \\
        T &= (x_{m+1} \vee x_{m+2} \vee \neg x_{m+2})
        \end{align}                
        $\phi_I$ represents the CNF of $I$. $I_h$ contains $2n+2$ clauses and $m+3$ variables. Note that clause $T = x_{m+1} \vee x_{m+2} \vee \neg x_{m+2}$ is always true. Then, we will prove $I$ and $I_h$ are equivalent.
        \begin{itemize}
        \item Suppose there exits an assignment of variables $x_i(i = 1,\dots,m)$ such that $\phi_I$ is $true$. If we assign $x_{m+1} = x_{m+2} = x_{m+3} = false$, which leads to $D=false$, $\phi_{Ih}$ will be separated into $n+1$ $false$ clauses(all $D$) and $n+1$ $true$ clauses($n$ clauses of $\phi_I$ and $T$). Thus, there exists an assignment such that exactly half the clauses of $\phi_{Ih}$ evaluate to false and exactly half the clauses of $\phi_{Ih}$ evaluate to true.
        \item Suppose there exists an assignment such that exactly $n+1$ clauses of $\phi_{Ih}$ evaluate to false and exactly $n+1$ clauses of $\phi_{Ih}$ evaluate to true. Since $T$ is always $true$, if $D=true$, there are at least $n+2 > n+1$ clauses with value $true$. This assignment is not possible under previous assumption. If $D=false$, there are $1$ $true$ clause($T$) and $n+1$ $false$ clause. Thus, the remaining $n$ clauses of $\phi_I$ must be $true$.
        \end{itemize}
        \item Thus, Half-3SAT problem is NPC.
        \end{proof}


\end{document}