\documentclass[]{article}

\usepackage{amsmath}
\usepackage{amsthm}
\usepackage{amssymb}
\usepackage{listings}
\usepackage{clrscode3e}
\usepackage{algorithm}
\usepackage{algorithmic}
\usepackage{graphicx}
\usepackage[top=2cm, bottom=2cm, left=2cm, right=2cm]{geometry}
\renewcommand\qedsymbol{$\blacksquare$}

\title{Algorithm Homework 2}
\date{2015-10-13}
\author{Jingwei Zhang 201528013229095}

\begin{document}
    \maketitle
    \section{Problem 1}
        \subsection{Sequence}
            \subsubsection{Optimal Substructure}
                \paragraph{}Suppose the houses are placed in a line from left to right labeled with integer from $1$ to $n$, each storing money $h_i(i = 1,\dots,n)$. The optimal substructure is the maximum amount of money $m_i$ the the robber can get from house $1$ to $i$($[1, i]$), the DP equation is:
                \begin{align*}
                m_i = \begin{cases}
                     0 \;, & \text{if } i=0 \\
                     h_i \;, &\text{if } i=1 \\
                     max(m_{i-1}, h_i + m_{i-2}) \;, & \text{otherwise}
                     \end{cases}
                \end{align*}
               \paragraph{}The answer to this problem is $m_n$.
            \subsubsection{Algorithm}
            \begin{codebox}
\Procname{$\proc{Maximim-Robbed-Money-Sequence}(H)$}
\li $n \gets \attrib{H}{length}$
\li $m[0] \gets 0$
\li $m[1] \gets H[1]$
\li \For $i \gets 2$ \To $n$
\li     \Do
            $m[i] = max(m[i-1], H[i] + m[i-2])$
        \End 
\li \Return $m[n]$
            \end{codebox}
            \subsubsection{Correctness}
                \begin{proof}
                $m[i]$ maintains the maximum amount of money the robber can get in house $[1,i]$. For $i = 0,1$, the correctness of $m[i]$ is obvious. Suppose $m[i]$ is correct $\forall i= 1,\dots,k-1$. For $i = k$:
                    \item If the robber steals house $k$, he will get $H[k]$ in it and have not steal house $k-1$; before house $k-1$ he gets at most $m[k-2]$, according to the optimal structure, $m[k] = H[i]+m[k-2]$. 
                    \item The correctness of this optimal structure: Suppose we have a smaller $m'[k-2](<m[k-2])$(not the optimal solution of sub-problem with size $k-2$) that generates the optimal solution $m'[k](>m[k])$ of larger problem with size $k$. Then we substitute its previous $k-2$ part with our optimal solution for sub-problem since house $k-1$ is not stolen, we get a new solution $m'[k] - m'[k-2] + m[k-2]$ for larger problem, which is larger than $m'[k]$, contradictory.
                 \item If he does not steal house $k$, no constrain exits on previous houses. Thus, $m[i] = m[i-1]$, according to the optimal structure.
                 \item The correctness of this optimal structure: Suppose we have a smaller $m'[k-1](<m[k-1])$(not the optimal solution of sub-problem with size $k-1$) that generates the optimal solution $m'[k](>m[k])$ of larger problem with size $k$. Then we substitute its previous $k-1$ part with our optimal solution for sub-problem since house $k$ is not stolen, we get a new solution $m'[k] - m'[k-1] + m[k-1]$ for larger problem, which is larger than $m'[k]$, contradictory.
                 \item Then, picking the maximum value of these two(steal or not) gets maximum amount of money the robber can get in house $[1,k]$, which means $m[i]$ is correct for $i = k$ and obviously when $i>n$ $\proc{Maximim-Robbed-Money-Sequence}$ stops. Thus the correctness of this algorithm is proven.
                \end{proof} 
            
            \subsubsection{Complexity}The size of this problem is the number of houses $n$. Thus, sequence $m$ has $n+1$ elements with $O(1)$ computing each. Thus the total time complexity is $O(n)$ and space complexity is    $O(n)$ for storing array $m$.
        
        
        
        \subsection{Circle}
            \subsubsection{Algorithm}
                \paragraph{}Suppose the houses are placed in a circle, and we arbitrary label one with integer $1$, then label others with integer from $2$ to $n$, clockwise, each storing money $h_i(i = 1,\dots,n)$.  Then, we enumerate all possible conditions of house 1, stolen or not stolen, then, the problem left is a sequence problem we have solved in previous part.
                \begin{codebox}
\Procname{$\proc{Maximim-Robbed-Money-Cirecle}(H)$}
\li $n \gets \attrib{H}{length}$
\zi \Comment If house $1$ is stolen, house $2$ and house $n$ must not be stolen.
\li $Hs \gets H[3,\dots,n-1] + H[1]$ 
\li $ms \gets \proc{Maximim-Robbed-Money-Sequence}(Hs)$
\zi \Comment If house $1$ is not stolen.
\li $Hns \gets H[2,\dots,n]$
\li $mns \gets \proc{Maximim-Robbed-Money-Sequence}(Hns)$
\li \Return $max(ms, mns)$
                \end{codebox}
            \subsubsection{Correctness}
            \begin{proof}
            If we enumerate two possible states of house 1, the left part ($[3,\dots,n-1]$ for stolen and $[2,n]$ for not stolen) have no connection directly between the head and tail house, thus, they are sequence problem. The correctness of sequence problem has been proven in previous section.
            \end{proof}
                
            \subsubsection{Complexity}
                \paragraph{} We call function $\proc{Maximim-Robbed-Money-Sequence}$ twice, both with size $O(n)$. This function costs $O(n)$ of time and $O(n)$ of space. Thus, the total time complexity is $O(n)$, total space complexity is $O(n)$.
        
    \section{Problem 2}
        \subsection{Optimal Substructure}
            \paragraph{}The optimal substructure is the minimum path sum $s_{i,j}$ from current place(row $i$, column $j$, $j \leq i$) to bottom, the DP equation is ($a_{i,j}$ denotes the number in row $i$ , column $j$):
            \begin{equation*}
            s_{i,j} = \begin{cases}
                     a_{i,j} \;,   & \text{if row i is the bottom row}   \\
                     a_{ij}+min(s_{i+1,j},s_{i+1,j+1}) \;, & \text{otherwise}
                     \end{cases}
            \end{equation*}
            \paragraph{} The answer to this problem is $s_{1,1}$.
        \subsection{Algorithm}
        \paragraph{Pseudo-code:} A is the matrix storing the number. r is the number of rows(columns) in A. S is the matrix storing the minimum path sum $S[i][j]$ from current place(row $i$, column $j$, $j \leq i$) to bottom.
\begin{codebox}
\Procname{$\proc{Minimum-Path-Sum}(A,r)$}
\li \For $j \gets 1$ \To $r$
\li     \Do
            $S[r][j] \gets A[r][j]$ 
        \End 
\li \For $i \gets n-1$ \To $1$
\li     \Do
            \For $j \gets 1$ \To $i$
\li             \Do
                    $S[i][j] = A[i][j] + min(S[i+1][j],S[i+1][j+1])$
            \End
        \End
\li \Return $S[1][1]$
\end{codebox}
        \subsection{Correctness}
        \begin{proof}[Proof of Optimal Substructure:]
        Suppose there exists a smaller path sum $s_l'$(what we get is $s_l$ in $\proc{Minimim-Path-Sum}$ and $s_l' > s_l$) in an arbitrary sub-problem(the min sum from a not-top place to bottom) which leads to the minimum global path sum $s_g'$(what we get is $s_g$ in $\proc{Minimum-Path-Sum}$ and $s_g' < s_g$). Due to the path from top to this place above and the path from this place to bottom are independent, we can always substitute $s_l'$ part with $s_l$ part, that leads a larger global path sum $s_g'' = s_g' + s_l - s_l' < s_g'$. However, this contradicts to the assumption that $s_g'$ is the minimum global path sum. 
        \end{proof}   
        \subsection{Complexity}
            \paragraph{}Let the size of this Problem be the total numbers in this triangle. $S$ matrix have totally $O(n^2)$ numbers with $O(1)$ for computing each. Thus, the total time complexity is $O(n)$ and space complexity is $O(n^2)$ for the storage of $S$.
            

    \section{Problem 5}
        \subsection{Optimal Substructure}
            \paragraph{}The optimal substructure is the number of ways ($w_j$) to decode the sequence $S[1, i]$ (suppose the original sequence is $S[1,n]$), the DP equation is:
            \begin{align*}
            w_j = \begin{cases}
                     0 \;,   & \text{if }j=0   \\
                     1 \;,   & \text{if }j=1   \\
                     w_{j-1} \;, & \text{if }j \geq 2 \text{ and } 10*S[j-1]+S[j] > 26 \\
                     w_{j-1} + w_{j-2} \;, & \text{if }j \geq 2 \text{ and } 10*S[j-1]+S[j] \leq 26 \\
                     \end{cases}
            \end{align*}
            \paragraph{}The answer to this problem is $w_n$.
        \subsection{Algorithm}
        \paragraph{Pseudo-code:}$S$ represents the message containing $n$ digits. $w$ stores the number of ways decoding the prefix of message sequence.
        \begin{codebox}
\Procname{$\proc{Number-of-Ways-Decoding-Message}(S)$}
\li $n \gets \attrib{S}{length}$
\li $w[0] \gets 0$
\li $w[1] \gets 1$
\li \For $j \gets 2$ \To $n$
\li     \Do
            \If $10*S[j-1]+S[j] > 26$
\li         \Then $w[j] \gets w[j-1]$
\li         \Else $w[j] \gets w[j-1] + w[j-2]$
            \End
        \End 
\li \Return $w[n]$
        \end{codebox}
        \subsection{Correctness}
            \begin{proof}
            $w[i]$ maintains the number of ways to decode the sequence $S[1,i]$. For $i = 0,1$, the correctness of $w[i]$ is obvious. Suppose $w[i]$ is correct $\forall i= 1,\dots,k-1$. For $i = k$:
                    \item If $10*S[k-1]+S[k] \leq 26$, which means $S[j-1]S[j]$ can be decoded as a single character. In this case the number of ways is $w[k-2]$ according to the optimal structure. Also, $S[k]$ can be decoded as a single character. In this case, the number is $w[k-1]$. Thus, $m[k] = w[k-1]+w[k-2]$.  
                    \item The correctness of optimal substructure: Suppose there exists a larger number of ways decoding the prefix sequence $S[1,k-1]$ with number $w_{k-1}'$(what we get is $w_{k-1}$ in $\proc{Number-of-Ways-Decoding-Message}$ and $w_{k-1}' \leq w_{k-1}$) which leads to the optimal global solution  $w_k'$(what we get is $w_k$ in $\proc{Number-of-Ways-Decoding-Message}$ and $w_k' \geq w_k$). If we substitute $w_{k-1}'$ with $w_{k}'$ we will get a larger optimal global solution. Contradictory. For sub-problem $k-2$, it is similar.
                    \item If $10*S[k-1]+S[k] > 26$, which means $S[j-1]S[j]$ can not be decoded as a single character. The only possibility for $S[j]$ is decoding it singly to a character. This leads to $w[i] = w[i-1]$ according to optimal structure mentioned above.
                    \item Thus, $w[i]$ is correct for $i=k$. Moreover, this procedure will stop after $w[n]$. Thus, the correctness of this algorithm is proven
            \end{proof}   
        \subsection{Complexity}
            \paragraph{}The size of this problem is the length of message sequence, $n$, the same as $w$. For each $w[j]$, computing costs only $O(1)$. So the total time complexity is $O(n)$ and space complexity is $O(n)$ for storing $w$.
    
    \section{Problem 6}
        \subsection{Algorithm}
            \paragraph{}Since there are two transactions(if there is only one, we can sell and buy the stock in a single day within this transaction), we can divide the problem into two independent sub-problems:the max profit $p_1$ of the first transaction within day $[0,i)$ and max profit $p_2$ of the second transaction within day $[i,n)$. We can enumerate all possible $i$ and find the max $p_1+p_2$. This costs $O(n)$.
            \paragraph{}To find the max profit in first transaction, we compute the max profit $ps_i$ we can get if we sell the stock in day $i$. The optimal substructure is the minimum price $min_i$ in day $[0,i]$, $min_{i+1} = min(p_i,min_i)$. Then $ps_i=p_i-min_i$, which costs $O(n)$ to enumerate all $i$ in $[0,n)$. Then the max profit $pm_i$ within day $[0,i)$ will be $pm_0 = 0,pm_i = max(pm_{i-1},ps_i)$, that costs $O(n)$.The second transaction is similar.
            \paragraph{}From what have mentioned above, the total time complexity is $O(n)$.
        \subsection{C++ Code}
            \lstinputlisting[language=C++]{stocks.cpp}
    
    
\end{document}