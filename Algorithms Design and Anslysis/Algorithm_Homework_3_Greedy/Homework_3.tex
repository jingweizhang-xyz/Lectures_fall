\documentclass[]{article}

\usepackage{amsmath}
\usepackage{amsthm}
\usepackage{amssymb}
\usepackage{listings}
\usepackage{clrscode3e}
\usepackage{algorithm}
\usepackage{algorithmic}
\usepackage{graphicx}
\usepackage[top=2cm, bottom=2cm, left=2cm, right=2cm]{geometry}
\renewcommand\qedsymbol{$\blacksquare$}

\title{Algorithm Homework 2}
\date{2015-10-13}
\author{Jingwei Zhang 201528013229095}

\begin{document}
    \maketitle
    \section{Problem 1}
        \subsection{Algorithm}
        \paragraph{}Non-decreasingly sort all vertex according to its degree. For every vertex $i$ link it to $d_i$ vertices following. If this forms a graph, then these vertices can form a graph, if not, they can not form a graph.
        \paragraph{Pseudo-code:} $D$ is the array storing the degree of every vertex. $\proc{Sort}(D)$ is a sorting procedure which sorts all integers in array $D$ non-decreasingly.
\begin{codebox}
\Procname{$\proc{Whether-Form-Graph}(D)$}
\li $n \gets \attrib{D}{length}$
\li $\proc{Sort}(D)$ 
\li \For $i \gets 1$ \To $n$ \label{Whether-Form-Graph:for i}
\li     \Do
            $j \gets i+1$
\li         \While $D[i] > 0$ and $j \leq n$ \label{Whether-Form-Graph:for j}
\li             \Do
                    \If $D[j] > 0$
\li                 \Then $D[j]--$
\li                       $D[i]--$
\li                       $j++$
                    \End
                \End
\li         \If $D[i] > 0$
\li         \Then \Return Can-Not-Form-Graph
            \End
        \End
\li \Return Can-Form-Graph
\end{codebox}
        \subsection{Correctness}
        \begin{proof}We will prove our strategy can get an undirected graph if and only if there exits a graph whose node degrees are precisely the numbers $d_1,d_2,\dots,d_n$.
        \item If degrees in $D$ can form a graph, non-decreasingly sort all vertex according to its degree. For each vertex $v_i$ with degree$d_i$ in non-decreasingly degree order, if it links to a vertex $v_j \notin v[i+1,i+d']$($d'$ denotes next $d'$ vertices that have exactly $d_i$ vertices linked to $v[i+1,n]$), then there must exits a vertex $v_k \in v[i+1,i+d']$ that links to $v_p, p > i$. Then rearrange the linkages from :$v_i-v_j,v_k-v_p$ to $v_i-v_k, v_j-v_p$. By such procedure, a new graph that following our strategy can be formed. Thus, our strategy can get an undirected graph.
        \item If our strategy can get an undirected graph, it is obvious that there exits a graph whose node degrees are precisely the numbers $d_1,d_2,\dots,d_n$.
        \end{proof}   
        
        \subsection{Complexity}
            \paragraph{}The program $\proc{Whether-Form-Graph}(D)$ contains one sorting functions. If we use quick sort for sorting function, the time complexity of sort is $O(n\log n)$. The $\For$ loop in line \ref{Whether-Form-Graph:for i} will run $n$ times, the $\For$ loop in line \ref{Whether-Form-Graph:for j} will run at most $n-i$ times. Totally $\sum_{i=1}^n{\sum_{j=1}^{n-i}O(1)} = O(n^2)$. 
            \paragraph{}Thus, total time complexity is $O(n^2)$. This algorithm's space complexity is $O(n)$ for storage of array $F$.            

    \section{Problem 4}
        \subsection{Algorithm}
        \paragraph{}We can prove that setting $a_i = a_{(i)}$ and $b_i = b_{(i)}$ will maximize the payoff. Thus, the algorithm is simply sort array $A$ and $B$ in non-decreasing order and summing all $a_i^{b_i}$.
        \paragraph{Pseudo-code:}$A,B$ are sets each containing $n$ integers. $\proc{Sort}(A)$ is a sorting procedure which sorts all integers in array $A$ non-decreasingly.
        \begin{codebox}
\Procname{$\proc{Maximum-payoff}(A,B)$}
\li $n \gets \attrib{A}{length}$
\li $\proc{Sort}(A)$
\li $\proc{Sort}(B)$
\li $payoff = 0$
\li \For $i \gets 1$ \To $n$
\li     \Do
            $payoff = payoff + A[i]^{B[i]}$
        \End 
\li \Return $payoff$
        \end{codebox}
        \subsection{Correctness}
            \begin{proof}
            Firstly we rearrange the elements in array $A$ and $B$ in non-decreasing order, then the problem is converted to equivalent one: all elements in $A$ is fixed, find a permutation of $B$ that maximize $p = \sum_{i=1}^n a_i^{b_i}$.
            \item Then we will prove that for any two elements $b_i$ and $b_j$ in $B$, the non-decreasing order of them will maximize the sum $a_i^{b_i} + a_j^{b_j}$. The mathematical representation is: $\forall a_i,a_j \in Z^+$, $\forall b_i, b_j \in Z^+$, if $a_i<a_j$ and $b_i < b_j$, then $a_i^{b_i} + a_j^{b_j} > a_i^{b_j} + a_j^{b_i}$. 
            \item According to $a_i<a_j$ and $b_i < b_j$, we will get:
            \begin{align*}
            & a_j^{b_i} > a_i^{b_i} \\
            & a_j^{b_j-b_i} > a_j^{b_j-b_i} \\
    \implies& a_j^{b_i}(a_j^{b_j-b_i}-1) > a_i^{b_i}(a_j^{b_j-b_i}-1) \\
    \implies& a_j^{b_j}-a_j^{b_i} > a_j^{b_j}-a_i^{b_i} \\
    \implies& a_i^{b_i} + a_j^{b_j} > a_i^{b_j} + a_j^{b_i}
            \end{align*}
\item Finally, we prove that the payoff $p^*$ generated by $B^*$, the non-decreasing order of $B$, is larger than payoff $p'$ generated by any other permutation $B'$ of array $B$. For any permutation $B' \neq B^*$, perform a bubble sort on $B'$ to change $B'$ to $B_s$ with $O(n^2)$ times of swapping adjacent increasing elements. Suppose this process performs altogether $m$ times of swap. According to the two elements situation above, each swap will generates a new sequence $B'_k(k = 1,\dots,m)$ with a greater payoff $p'_k(k = 1,\dots,m)$. After $m$ times of swap, we get $B'_m = B^*$ with payoff $p'_m = p^* > p'$.
            \end{proof}   
        \subsection{Complexity}
            \paragraph{}The program $\proc{Maxium-Payoff}(A,b)$ contains two sorting functions and a scan from $1$ to $n$. If we use quick sort for sorting function, then its time complexity is $O(n\log n)$.
    
    \section{Problem 6}
        \subsection{Result}
            \paragraph{} The program is compiled using g++ with -O3 parameter. Blow is the running time of Dijkstra's shortest path algorithm using different data structures. It runs on graph.txt with $20000$ nodes. Algorithm starts from node $6676$ to node $18853$ and returns shortest path length $29$.
            \begin{table}[h!]
            \centering
            \caption{Running Time of Dijkstra's Shortest Path using Different Data Structures}
            
            \begin{tabular}{l|*{4}{c}}
            \hline
Data structure & Linked List & Binary Heap & Binomial Heap & Fibonacci Heap \\
\hline
Running time(second) & $1.22607$ & $0.000602$ & $0.027132$ & $0.018106$ \\
            \hline            
            \end{tabular}
            \end{table}
        \subsection{C++ Code}
            \lstinputlisting[language=C++]{Dijkstra.cpp}
    
    
\end{document}